\label{LiquidLib}

With the recent increase in computational power, molecular dynamics (MD) simulations have emerged as a powerful tool for understanding and comparing to experimental results. MD and MC simulations have provided great insight across engineering and science fields alike, and have been applied to materials ranging from inorganic compounds like metals to biological systems such as protein and DNA. MD and MC simulations are especially imperative to the study of liquids and liquid-like systems, as they provide understanding to the structural and dynamical behavior of liquid systems in and out of equilibrium.  Similarly, researchers use MD and MC simulations to understand and interpret results measured by neutron scattering experiments.

These simulations result in trajectories that provide the positions, $\mathbf{r}_i(t_j)$, and the velocities, $\mathbf{v}_i(t_j)$, of the atoms, where the subscript $i$ refers to the $N$ different atoms in the system and the subscript j refers to the $M$ time steps. Simulations result in $6NM$ data values, which, to obtain meaningful results, need to be reduced to conceptually simpler, useful, and in many instances, measurable quantities. Reduction of these simulations to simpler and understandable quantities requires programs to perform these analyses.

Numerous open source and commercial packages exist to perform molecular dynamics and Monte Carlo simulations, but outside of a few simple quantities such as pair distribution function or mean squared displacement, these packages often do not offer the option to compute quantities needed to interpret the simulation or compare to experimental results for liquid systems. Thus, most researchers are forced to write in-house scripts and programs to compute the desired quantity, but these tend to only be useful to the respective researcher. On the other hand some codes can be obtained online to compute select quantities, but these are scattered, requiring users to seek out multiple packages, learn the code's implementation, and possible convert their trajectory to match the file type required by the package. As a result, a few attempts to create a comprehensive post-processing package for molecular dynamics simulations emerged, such as the open source software TRAVIS \cite{TRAVIS} or nMOLDYN \cite{nMOLDYN}. 

We introduce LiquidLib, an open-source and comprehensive package for post-processing of molecular dynamics and Monte Carlo simulations of liquids for comparison to neutron scattering experiments \cite{LiquidLib}.  LiquidLib was developed in the same spirit as TRAVIS \cite{TRAVIS} and nMOLDYN \cite{nMOLDYN} while greatly expanding on the capabilities of the packages.  LiquidLib extends the list of calculable quantities, the list of readable file types while also offering an easy method to add a file type if not already offered, and implements efficient parallelized C++ code to perform analysis in speeds unmatched by similar packages.  The goal of LiquidLib is to create a singular package to compute all desired quantities for studying liquid simulations. Further, LiquidLib is easily expandable to be able to read a simulation trajectory from any program. At the moment, LiquidLib can read trajectories from LAMMPS (.xyz, .xtc, .dump, .atom) \cite{LAMMPS}, GROMACS (.xtc, .trr, .gro) \cite{GROMACS}, and VASP (XDATCAR) \cite{VASP}, however we hope to expand on this list. The variety of trajectory data supported by LiquidLib allows users to run computations for systems with different length scales including atoms, molecules, colloids, coarse-grained models, and etc.  LiquidLib also contains a corresponding graphical interface to aid in the construction of input scripts and the analysis procedure.  LiquidLib's list of quantities currently includes the following, the pair distribution function, the static structure factor, the mean squared displacement, the non-Gaussian parameter, the four-point correlation, the velocity auto correlation, the self and coherent van Hove correlation, the self and coherent intermediate scattering functions, and the bond orientational order parameter.  LiquidLib offers the ability to weight quantities based on the neutron scattering lengths for multi-component systems, which allows the results to be comparable to neutron scattering experiments. 

While not extensively used, numerous studies into the behavior of colloids, liquids, and glasses with hyper-dimensionality have been performed in recent years \cite{Charbonneau2012, Charbonneau2013}. Similarly, we have also created an in-house package that simulates Lenard-Jones liquids in high dimensionality, which requires special care to analyze the trajectories in high dimensions. As a result, LiquidLib differs from other reduction packages in that the dimension of the simulation can be set to any positive integer and the quantities will be correctly calculated.  While LiquidLib was written for the context of liquids and liquid-like systems with comparison to neutron scattering experiments, the quantities calculable by LiquidLib are useful for non-liquids systems and comparison to analyses other than neutron scattering. Simple extension to compare with X-ray measured experiments corrected for atomic-form factors can be performed by unweighted computations whereby the scattering lengths are set to unity.

For the purpose of this project, LiquidLib was extended to incorporate the ability to read MTD trajectories from GROMACS.  Further, LiquidLib was extended to compute the bond orientational order parameter.  LiquidLib was developed as a side project to this project for the main purpose of MD simulations and extended to aid metadynamics simulations.  LiquidLib is developed and maintained by the Zhang Research Group, particularly at this time, by Zhikun Cai, Abhishek Jaiswal, and Nathan Walter \cite{LiquidLib}.