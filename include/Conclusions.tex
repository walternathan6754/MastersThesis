\section{Conclusions}
In this thesis, we applied the recently successful energy landscape framework and metadynamics method to study nucleation and crystal growth, and compared the energy landscapes of a fragile glass former and a good crystal former to verify the energy landscape framework with the metadynamics method.  Traditional molecular dynamics is hindered by temporal limitations and Monte Carlo lacks dynamical information prohibiting these two methods from capturing the nucleation and crystal growth process fully.  As a response, we successfully implemented metadynamics into the open source package GROMACS with full OpenMP parallelization.  Metadynamics simulations allowed us to calculate the free energy barrier as a function of temperature involved in the nucleation and crystal growth process.  Unlike many other methods to compute this free energy barrier, metadynamics requires no \textit{a priori} assumptions for the calculation of the free energy barrier.  Thus, we believe that the use of metadynamics is more accurate for the calculation of this energy barrier and non-monotonic nucleation rate. 

Herein, we applied the metadynamics method to a model liquid monotonic argon system, and sampled the energy landscape.  From the potential energy landscape, we were able to determine the energy barriers involved in the nucleation and crystal growth process.  We determined that there are two unique energy barriers involved in the process, and each one dominates in a different regime.  At high subcoolings, the crystallization process is diffusion limited, and the crystal growth energy barrier dominates the crystallization process.  At low subcoolings, the crystallization is kinetically limited, and the formation of nucleation sites limits the crystallization process.  We find a non-monotonic temperature dependence of the overall crystallization energy barrier and rate.  Further, we apply metadynamics to a binary Lennard-Jones system, which is a known good glass former.  By comparing the monotonic Lennard-Jones system to the binary Lennard-Jones system, we are able to see the stark difference between the energy landscapes during crystallization and during glassy dynamics.

\section{Future Work}
Thus far, we have had great success with the metadynamics method.  The method was successfully built into the open source package GROMACS, with the addition of dynamic minimization step sizing and OpenMP parallelization.  Also, the method was successfully used to study nucleation in a LJ system and used to computationally confirm the differences in the energy landscape between good glass formers and good crystal formers.  However, a great deal of future work still exists.  The three categories of future work can be summarized into further application of the method to new systems, further analysis of the simulations, and development of enhanced metadynamics methods.

\begin{itemize}
	\item \textbf{Applications/systems}  
	In this thesis, we demonstrated two important applications of the metadynamics method.  We applied the metadynamics method to study crystallization in a monoatomic Lennard-Jones system, and also applied the method to study a fragile glass forming binary Lennard-Jones for comparison to the monoatomic counter part.  Lennard-Jones is a great starting potential because it is very well studied, simple to apply, and simple to understand. In particular, Lennard-Jones is known to exhibit an FCC crystal structure.  However, we wish to extend our work by applying the metadynamics method developed herein to more complex potentials and more application relevant materials (Lennard-Jones is not real-world common).  
	
	There are several systems we wish to extend this analysis to.  First, we wish to study ST2 water with this method.  We have already begun simulations of this system (both MD simulations and MTD simulations).  The full details of our preliminary results of the model is contained in Appendix \ref{ST2}.  The ST2 model and water in general is known to have several crystal polymorphs, exhibit supercooled behavior, and is critical in an application sense \cite{Haji-Akbari2015}.  We hope the metadynamics method may elucidate why one polymorph is dominate at different points in the phase diagram.  Another potential application is to apply metadynamics to a thin film system.  This would allow us to study surface behavior of a system.  We also wish to apply the method to protein systems in order to study the folding-unfolding process which is macroscopically magnitudes longer than MD can capture.  Lastly, the possibility to study metallic systems is available if we extend the method to LAMMPS \cite{LAMMPS}, another open source MD package known for studying metallic systems.
	
	Another study we wish to perform is the study of the landscape and the effect on fragility of glass formers.  In this thesis, we already provided insight to the differences between fragile glass formers and poor glass formers.  We wish to extend this to strong glass formers and fragile glass formers of varying degree.  Then, we can find a connection between the landscape statistics, roughness, and other landscape quantities to the glass formers fragility.  This could offer insight or at least provide a theoretical explanation for the difference in glass formers behavior, and potentially offer a predictive method for estimating new glass formers fragility.
	
	Lastly, we wish to apply metadynamics to other known macroscopically long timescale events, such as material aging and  degradation.  Both of these events can occur on orders of magnitude longer timescales than accessible to molecular dynamics simulations.  By using metadynamics, we can observe the material aging and degradation process from an atomic scale.  
	
	\item \textbf{Further Analysis}
	
	%% determine J_o
	Metadynamics simulations, like MD and MC simulations, requires a great deal of analysis to extract useful information from the simulation.  While this report various analysis of a MTD simulation, there is still a great deal of information hidden in the simulation that requires extraction.  First, we have successfully computed the potentially energy barriers involved in the nucleation process and estimated the time scale involved in the process.  However, in order to fully benchmark our results to experiments we need to compare to known nucleation and crystallization rates, which requires us developing a method to compute $J_o$, the scaling factor for the nucleation rates we calculated.  We wish to estimate this value from the simulation and compare to experiments, rather than extract the value from experiments.
	
	%% determine frequency distributions to compare to REMA
	Recent work by Cai et. al. showed that they could extract the activation barriers and frequencies from simulations and experiments via a transform of the dynamic structure factor \cite{CaiREMA}.  We have thus far been able produce comparable activation barrier distributions as a function temperature for the same binary Lennard-Jones system \cite{CaiRMA}.  We wish to extend our analysis to be able to also reproduce their frequency distribution results.  Metadynamics is capable of providing the activation barrier information by sampling the energy landscape, however, our method does not contain true dynamics making calculation of the frequencies difficult at this moment.  We hope to calculate the frequencies by in parallel with the Metadynamics simulations run MD simulations with small energies to calculate the frequency distributions.
	
	%% compare to insitu diffraction patterns
	Experimental studies of nucleation is widely known.  However, recent work with an electro static levator and X-ray/Neutron scattering diffraction experiments have given hope that experimental investigation of the structural changes during nucleation maybe possible.  The electro static levator suppresses the heterogeneous nucleation in a sample and scattering experiments provide a probe to measure atomic structure.  As long as the flux is sufficiently high, time resolved inspection of homogeneous nucleation might be possible and would allow direct comparison to our simulations.
	
	\item \textbf{Enhanced Methods}
	
	While the metadynamics method has already shown great success, it can still be greatly improved upon.  The strength of the metadynamics method is dependent on the number of minimums and saddles points sampled, which is dependent on the number of penalties applied during the simulation.  As shown previously in this report, as more penalties are applied the time for the system to converge to a new minimum linearly increases due to the increased computational cost of computing the forces in the system.  Therefore, in order to study large systems such as proteins, we need to sample a large amount of barriers in an efficient amount of time.  As a result, we are pursuing a few new methods for improving on the sampling methods used in this report.  
	
	We have already tried a few enhanced methods, which are shown in depth in Appendix \ref{enhanced methods}.  Briefly, we attempted two methods of variable reduction.  One way to increase the speed of the metadynamics is to decrease the number of reaction coordinates.  However, determining which reaction coordinates are relevant and which are extraneous is a nontrivial task.  We developed two methods, bond order restrained metadynamics and bond order restrained with dynamic unfreezing metadynamics, to determine which variables are extraneous.  The two methods used the bond order parameter to determine which atoms were considered structured and would freeze these atoms, thus reducing the number of forces computed on each iteration.  The dynamic freezing option allowed atoms to become unfrozen if the bond order parameter fell below a threshold value.  While these methods increased the number of penalties applied to the system, the method also removed the variables necessary for crystal growth to occur.  The results of these methods are summarized in Appendix \ref{enhanced methods}  
	
	Another method we attempted for enhanced sampling was coined ``steepest ascent,'' which inverts Newton's steepest descent method.  The goal of the method was to use steepest descent for energy minimization to converge on a local minimum, then to use steepest ascent to converge on a local maximum or saddle point.  However, the method's down side is it is highly unstable and often converges to global maximum (infinite potential energy) rather than converging to the low energy saddle points.  We further attempted introducing methods to prevent the method from converging on infinite energy maximum, but saw little success yet. The results of this method are fully explained in Appendix \ref{enhanced methods}.  
	
	Even though these methods have not produced concrete results as of yet, we plan to continue developing these methods as the foundation for a more robust and efficient metadynamics method.  We believe with further work and development of the ideas behind these methods then when they come to full fruition, the methods will produce viable results in a far more effective and efficient manner than the metadynamics method used in this report.
\end{itemize}