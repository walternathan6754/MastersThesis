%% ABSTRACT Material
A material's properties can be effected greatly by the method of synthesis of the material.  The first step in the synthesis of any material is the nucleation of the material's phase.  Nucleation and crystal growth are understood to be activated processes involving the crossing of free-energy barriers. Attempts to capture the entire crystallization process over long timescales with molecular dynamic simulations have met major obstacles because of the temporal constraints of molecular dynamics. Herein, we circumvent this temporal limitation by using an improved metadynamics method based on the adaptive basin-climbing algorithm to directly sample the potential-energy landscape of a monotonic, model-liquid Argon system. The algorithm biases the system to evolve from a liquid-like structure towards an FCC crystal structure through inherent structures.  Compared to other computational techniques, our method requires no assumptions about the shape, size, or thermodynamics properties of the critical crystal nucleus, and does not require a nucleation seed to simulate the growth process.  Consequently, the sampled timescale is macroscopically long, magnitudes longer than molecular dynamics simulations.  Thus, we observe that the formation of a crystal involves two processes, each with a unique temperature-dependent energy barrier. One barrier corresponds to the creation of a crystal nucleation site; the other barrier corresponds to the crystal growth. We find the two processes dominate in different temperature regimes.  Further, we provide empirical evidence for the non-monotonic temperature dependence of the nucleation energy barrier and the nucleation rate.  Then, we also use metadynamics on a fragile glass forming system, and compare the landscape and timescale of the fragile glass former to the good crystal former.  The stark difference in landscapes provides an energy landscape explanation for the nucleation process.  The success of this method on a model liquid system is encouraging for elucidating the crystallization of more complex systems. 